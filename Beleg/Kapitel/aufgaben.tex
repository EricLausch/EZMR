\newpage
\chapter{Programmieraufgaben}
\vspace{-15 pt}
\section{Sockets}
\vspace{-15 pt}
Ein Socket ist eine Softwarestruktur, die zur Netzwerkkommunikation verwendet wird. Darüber hinaus werden Sockets auch zur Interprozesskommunikation eingesetzt. Darauf wird in der zweiten Aufgabe detaillierter eingegangen. Sockets sind bidirektional und der jeweilige Endpunkt des Kommunikationskanals. Über diesen Kanal können Anfragen gesendet und auch die entsprechenden Antworten empfangen werden. Client und Server besitzen einen eigenen Socket, dieser besteht aus Ziel- bzw. Quell-IP-Adresse, Ziel- bzw. Quellport sowie dem zu verwendenden Protokoll\cite{2009:rhein}.
\subsection{Teilaufgabe A}
\vspace{-15 pt}
\subsection{Teilaufgabe B}
\vspace{-15 pt}
\subsection{Teilaufgabe C}
\vspace{-15 pt}
\subsection{Teilaufgabe D}
\vspace{-15 pt}

\section{Interprozesskommunikation}
\vspace{-15 pt}
\subsection{Teilaufgabe A}
\vspace{-15 pt}
\subsection{Teilaufgabe B}
\vspace{-15 pt}
\subsection{Teilaufgabe C}
\vspace{-15 pt}
\subsection{Teilaufgabe D}
\vspace{-15 pt}