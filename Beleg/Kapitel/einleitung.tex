\newpage
\chapter{Einleitung}
\vspace{-15pt}
Mobile Roboter sind den meisten Menschen wohl in Form von Putzrobotern geläufig. In der Industrie werden sie ebenfalls weit verbreitet eingesetzt. Eine häufige Anwendung für mobile Roboter ist der Warentransport innerhalb der Produktion. Es können beispielsweise Waren aus Lagern geholt, in der Fertigung Teile an weiterverarbeitende Stationen geliefert oder auch wie im Heimbereich Reinigungsarbeiten durchgeführt werden. Der Unterschied zu einem Fahrerlosen Transportsystem ist die flexible Wegfindung der autonomen mobilen Roboter. Diese bewegen sich nicht auf festen vorgegebenen Pfaden sondern finden innerhalb ihrer Bewegungszonen den optimalen Weg zu ihrem Ziel. Mit Hilfe von Sensoren und Technologien wie Radar, erfolgt die Orientierung im Raum aber auch die Hinderniserkennung. Die mobilen Roboter umfahren im Weg befindliche Hindernisse selbständig. Die Verwaltung der gesamten in einem Betrieb vorhanden Roboter kann mit Software erfolgen. Dort können beispielsweise bestimmte Aufgaben zugewiesen und priorisiert werden \cite{2025:omron}.

In diesem Beleg sollen zunächst einige Grundkonzepte, die bei Echtzeitsystemen Anwendung finden anhand von Beispielen umgesetzt werden. Dazu gehören Sockets und die Interprozesskommunikation. Anschließend werden die Grundlagen des Roboter Betriebssystemes ROS mit Hilfe eines Versuchsroboters angewendet. Im Rahmen einer kleinen Beispielanwendung soll das Verständnis der Funktionsweise von ROS belegt werden. Abschließend erfolgt die Erprobung der erlangten Erkenntnisse mit einem professionellen mobilen Roboter.