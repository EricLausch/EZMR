
\usepackage[utf8]{inputenc}						% zur Verwendung von Umlauten
\usepackage[ngerman]{babel}						% für Silbentrennung
\usepackage[ngerman]{translator}			    % Mechan. zum Umsetzen sprachabh. Begriffe

\usepackage{amsmath}                            % für bestimmte mathematische Funktionen
\usepackage{amssymb}                            %
\usepackage{ziffer}                             % für intelligente Leerzeichenregelung nach Kommas in der Matheumgebung

\usepackage[pdftex]{graphicx}					% für Grafiken
\usepackage{subfigure}						% Bilder nebeneinander anordnen
\usepackage{wrapfig}                            % für in Fließtext eingebundene Abbildungen und Tabellen
\usepackage{subcaption}
\usepackage{caption}


\usepackage{ulem}                               % Schrift hervorheben
\usepackage{color} 								% Schrift färben

\usepackage{chngcntr}
\counterwithout{figure}{chapter}
\counterwithout{figure}{section}

\useunder{\uline}{\ul}{}                        % für Tabellen von http://www.tablesgenerator.com/#

\usepackage{pdfpages}							% bindet PDFs ins Dokument ein

\usepackage[bookmarksnumbered=true, colorlinks=false, pdfborder={0 0 0}]{hyperref}			% für Verlinkungen
\usepackage[figure]{hypcap}					    %
\hbadness=100000

\usepackage[onehalfspacing]{setspace}           % Zeilenabstand von 1
\usepackage[left=25mm, right=25mm, top=25mm, bottom=35mm]{geometry}
\renewcommand*{\chapterheadstartvskip}{\vspace*{0cm}}
\renewcommand*{\chapterheadendvskip}{\vspace{0cm}}



\usepackage[nonumberlist,acronym,toc]{glossaries}   % für Abkürzungsverzeichnis
\renewcommand*{\glspostdescription}{}				% Den Punkt am Ende jeder Beschreibung deaktivieren
\deftranslation[to=German]{Acronyms}{Abkürzungsverzeichnis}
%\newglossary[slg]{symbolslist}{syi}{syg}{Symbolverzeichnis}	%Ein eigenes Symbolverzeichnis erstellen
\makeglossaries										% Glossar-Befehle anschalten
%\loadglsentries{Abkuerzungsverzeichnis.tex}

\usepackage[final]{listofsymbols}               % für Symbolverzeichnis

\usepackage{scrwfile}                           % für Anhangsverzeichnis
\TOCclone[\contentsname~(\appendixname)]{toc}{atoc}
\newcommand\StartAppendixEntries{}
\AfterTOCHead[toc]{%
  \renewcommand\StartAppendixEntries{\value{tocdepth}=-10000\relax}%
}
\AfterTOCHead[atoc]{%
  \edef\maintocdepth{\the\value{tocdepth}}%
  \value{tocdepth}=-10000\relax%
  \renewcommand\StartAppendixEntries{\value{tocdepth}=\maintocdepth\relax}%
}
\newcommand*\appendixwithtoc{%
  \cleardoublepage
  \appendix
  \addtocontents{toc}{\protect\StartAppendixEntries}
  \listofatoc
}

\usepackage{scrlayer-scrpage}                   % für Kopf- und Fußzeile
\clearpairofpagestyles
\definecolor{gray}{rgb}{0.5,0.5,0.5}
\newcommand{\chaptercolor}{gray}
\addtokomafont{headsepline}{\color{gray}}
\addtokomafont{footsepline}{\color{gray}}
\renewcommand*{\chaptermarkformat}{}
\automark[chapter]{chapter}
\ohead{\color{\chaptercolor}\headmark}
\ofoot*{\color{\chaptercolor}\pagemark}
